\documentclass[11pt,nofootinbib]{report}
\usepackage[T1]{fontenc}
\usepackage[utf8]{inputenc}
\usepackage{authblk}
\usepackage{hyperref}
\usepackage{booktabs}
\usepackage{enumitem}
\usepackage{geometry}
\usepackage{graphicx}
\usepackage{subfiles}
\usepackage[utf8]{inputenc}
\usepackage[english]{babel}
\usepackage[utf8]{inputenc}
\usepackage{cite}
\usepackage{mathtools}
\usepackage{baththesis}
\usepackage{subcaption}
\usepackage{array}
\newcolumntype{P}[1]{>{\centering\arraybackslash}p{#1}}
\usepackage{booktabs,makecell}
\usepackage{color} 
\usepackage[dvipsnames]{xcolor}
    


\title{Accessible Full-Waveform LiDAR Data for Forest Monitoring}
\author{Milto Miltiadou}
\degree{Doctor of Engineering}
\department{Centre for Digital Entertainment}
\departmentPML{\\NERC Airborne Research Facility}
\degreemonthyear{November 2016}

\norestrictions


\begin{document}
	\maketitle
 		
\begin{abstract}	
     no more than 300 words
     
     NOTES:
     \par {\color{blue}{Blue colour: additions according to Neill's feedback}},
     \par {\color{red}{Red colour: notes}}
     \par {\color{gray}{Gray colour: text that is going to be modified }}
     
     \thispagestyle{empty}
\end{abstract}

\newpage
\thispagestyle{empty}
	\setcounter{secnumdepth}{0}
	\pagenumbering{roman}
	To be added on top
	\section{Abstract}\label{Abstract}
		\subfile{tex/Abstract.tex}
		\thispagestyle{empty}
		\newpage
	\section{Acknowledgements}\label{Acknowledgments}
		\subfile{tex/Acknowledgements.tex}
		\thispagestyle{empty}
		\newpage
	

	\section {Abbreviations and Glossary}\label{Abbreviations}
	\subfile{tex/Abbreviations.tex}
	\newpage
	\section{Publications}
		\textbf{DASOS-User Guide}, M. Miltiadou, N.D.F Campbell, M. Brown, S.C. Aracil, M.A. Warren, D. Clewley, D.Cosker, and M. Grant, Full-waveform LiDAR workshop at Interpine Group Ltd, Rotorua NZ, 2016\newline
		\textbf{Alignment of Hyperspectral Imagery and Full-Waveform LiDAR data for visualisation and classification purposes}, M. Miltiadou, M. A. Warren, M. Grant, and M. Brown, \textit{The International Archives of Photogrammetry, Remote Sensing and Spatial Information Sciences}, vol. 40, no. 7, p. 1257, 2015.\newline
		\textbf{Reconstruction of a 3D Polygon Representation from Full-Wavefrom LiDAR data}, M. Miltiadou, M. Grant, M. Brown, M. Warren, and E. Carolan,\textit{RSPSoc Annual Conference, New Sensors for a Changing World}, 2014.\newline
	 
	
	\section{Awards}
		\textbf{EDE and Ravenscroft Prize - Finalist}: Selected as one of the five finalists for this is a prestigious prize that recognises the work of best postgraduate researchers.\newline
		\textbf{Student Poster Competition} at Silvilaser.
		
	
	\section{Conference Presentations}
		\textbf{ForestSAT Conference}, Santiago, Chile, 2016 - Oral and Poster Presentation \newline
		\textbf{Computer Graphics \& Visual Computing (CGVC)}, Bournemouth, United Kingdom, 2016 - Poster Presentation \newline
		\textbf{Silvilaser}, La Grant Motte, France, 2015 - Poster Presentation \newline
		\textbf{International Symposium of Remote Sensing of the Environment 	(ISRSE)}, 2015 - Oral Presentation\newline
		\textbf{RSPSoc Conference}, New Sensors for a Changing world, Aberystwyth, United Kingdom, 2014 - Oral Presentation \newline
	
	
	

    \tableofcontents	 
		\newpage  
	\cleardoublepage
	% \phantomsection
	\addcontentsline{toc}{chapter}{\listfigurename}
	\listoffigures
	
	
		
		
	\setcounter{secnumdepth}{4}		

	\chapter{Introduction} \label{Introduction} 	\pagenumbering{arabic}
		\section{Forest Monitoring: Importance and Applications}\label{sec:forestMonitoring}
			\subfile{tex/ForestMonitoring.tex}
		\section{Background Information about Remote Sensing and Airborne Laser Scanning Systems}\label{Background}
			\subfile{tex/Background.tex}
			\newpage
		
    \chapter{Acquire Data}\label{AcquireData}
	    \subfile{tex/AcquireData-mike.1.tex}
	    \newpage
	\chapter{Overview of Thesis}\label{PipeLine}
		\subfile{tex/Overview.tex}
	\chapter{DASOS}\label{DASOS}
		\subfile{tex/DASOS.tex}	
		DASOS can produce 3D polygons and metrics useful in deriving information about the scanned areas
		
	\chapter{Voxelisation}\label{Voxelisation}
	    \subfile{tex/Representations.tex}
	    \newpage

		 		
	\chapter{Visualisations}\label{Visualisations}
		\subfile{tex/Visualisations.tex}
		\newpage
		\section{Integral Volumes}\label{Rep_IntegralVolumes}
			 \subfile{tex/Representations/IntegralVolumes.tex}
			 \newpage
	\chapter{Alignment with Hyperspectral Imagery}\label{Alignment}
		\subfile{tex/Alignment.tex}
		\newpage		
	\chapter{Classifications}\label{Classifications}
		\subfile{tex/Classifications.tex}
		\newpage
	\chapter{Comparison with Discrete Data}\label{ComparisonDiscreteVsFW}
	    \subfile{tex/Comparison.tex}
	    \newpage
	\chapter{Overall Results}\label{Results}
		\subfile{tex/OverallResults.tex}
		\newpage
	\chapter{Conclusions}\label{Conclusions}
		\subfile{tex/CoclusionsFutureWork.tex}
		\newpage
		\section{Contributions}\label{Contributions}
		\subfile{tex/Contributions.tex}
		
		\addcontentsline{toc}{chapter}{Bibliography}
	    \bibliographystyle{ieeetr}
	    \bibliography{mybib}{}

		\newpage
	\chapter{Appendices}\label{Apendices}
		\section{Birds and Mammals Catalogue} 
		\subfile{tex/AppendixBirds.tex}\label{AppendixBirds}
		\newpage
	
		
	
    
\end{document}
