\documentclass{subfiles}

% {\color{blue}{We can change colors for emphasis}}

\begin{document}

Forest monitoring {\color{blue} involves checking and observing the changes in the structure of the forests and their foliage over the years. It has a significant value in both sustainable and commercial forests, because it contributes} into managing biodiversity, maintaining forest health and optimising wood trade procedures as explained below: 
\begin{itemize}
 \item \textbf{Biodiversity} plays a substantial role in ecosystem resilience \cite{Elmqvist2003} while various human activities affect biological communities by altering their composition and leading species to extinction \cite{Hooper2005}.  For example, in Australian native forests many arboreal mammals and birds rely on hollow trees for shelters \cite{Lindenmayer2010} {\color{blue}. Hollow trees are trees that have hollows, which are semi-enclosed cavity on trunks and branches. They are formed by natural forces, like bacteria, fungi and insects and it takes hundreds of years to become suitable for animal/bird shelters. Studies} predict shortage of hollows available for colonisation in near future \cite{Goldingay2009}\cite{Gibbons2002}. Therefore monitoring  and protecting hollow trees have a positive impact in preserving biodiversity.
 
 \item \textbf{Forest Health}: Protecting vegetation from pests and diseases. An example of pests are the Brushtail Possums, which were initially brought to New Zealand for fur trade, but they have escaped and become a thread to native forests and vegetation ~\cite{DepartementOfConversation2014}. In addition, anthropogenic factors have a negative impact to nature. For instance, acid rain is responsible for the freezing decease at red bruces because it reduces the membrane-associated calcium, which is important for tolerating cold 
 \cite{DeHayes1999}. Those changes in nature need to be monitor in order to preserve a healthy and resilience ecosystem. 

 
 \item \textbf{Wood Trade}:  Measuring stem volume and basal areas of trees contributes to forest planning and management \cite{Holmgren2004}. For example, measuring stocking and wood quality would help into estimating the cost of harvesting the trees in relation to the stocking \cite{Susana2015}.
 
\end{itemize}
 
{\color{blue}Traditionally, forest monitoring involves field work such as travelling into the area of interest and measuring tree diameters and heights. Regarding the need to monitor hollows}, tree climbing with ladders and ropes gives very accurate results but it's dangerous, expensive, time consuming, and cannot easily scale into large forested areas \cite{Harper2004}\cite{Rayner2011}. Therefore, automated ways of monitoring forests are essential and this is why Remote Sensing has a significantly positive impact in forestry. 
 
\end{document}