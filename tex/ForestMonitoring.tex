\documentclass{subfiles}

\begin{document}

Forest monitoring has a significant value in both sustainable and commercial forests and it could contribute into managing biodiversity, maintaining forest health and optimising wood trade procedures as explained below: 
\begin{itemize}
 \item \textbf{Biodiversity} plays a substantial role in ecosystem resilience \cite{Elmqvist2003} while various human activities affect biological communities by altering their composition and leading species to extinction \cite{Hooper2005}.  For example, in Australian native forests many arboreal mammals and birds rely on hollow trees for shelters \cite{Lindenmayer2010} and studies predict shortage of hollows available for colonisation in near future \cite{Goldingay2009}\cite{Gibbons2002}. Therefore monitoring those native forests and protecting hollow trees will have a positive impact in preserving biodiversity.
 \item \textbf{Forest Health}: Protecting vegetation from pests and diseases. An example of pests are the Brushtail Possums, which were initially brought to New Zealand for fur trade, but they have escaped and become a thread to native forests and vegetation ~\cite{DepartementOfConversation2014}. In addition, anthropogenic factors have a negative impact to nature. For instance, acid rain is responsible for the freezing decease at red bruces because it reduces the membrane-associated calcium, which is important for tolerating cold 
 \cite{DeHayes1999}. Those changes in nature need to be monitor in order to preserve a healthy and resilience ecosystem. 

 
 \item \textbf{Wood Trade}:  Measuring stem volume and basal areas of trees contributes to forest planning and management \cite{Holmgren2004}. For example, measuring stocking and wood quality would help into estimating the cost of harvesting the trees in relation to the stocking \cite{Susana2015}.
 
\end{itemize}
 
Traditional ways of monitoring forest talks about fieldwork. Fieldworks implies travelling into the area of interest and taking measurements like tree diameter and height. Regarding the hollows monitoring necessity, tree climbing with ladders and ropes gives very accurate results but it's dangerous, cost expensive, time consuming, and cannot easily scaled into large forested areas \cite{Harper2004}\cite{Rayner2011}. Therefore, automated ways of monitoring forests are essential and this is why Remote Sensing has a significantly positive impact in forestry. 
 
\end{document}