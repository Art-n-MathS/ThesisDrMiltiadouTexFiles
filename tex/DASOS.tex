\documentclass{subfiles}

\begin{document}
	
	\par As mentioned at Section \ref{sec:Problem}, there are very few uses of FW LiDAR data because of the quantity and size of the stored information. For that reason, DASOS was developed DASOS (Section \ref{DASOS}), as an open source software, to help foresters without computer science background to use FW LiDAR data. In this section:
	
	\begin{itemize}
		\item An overview of related software packages is given and it is explained how DASOS differs from those packages (Section \ref{LiDARsoftwares}).
		\item  The main way of interpreting the data within DASOS (the voxelisation approach) is described (Subsection \ref{Voxelisation}).
		\item Then all the functionalities of DASOS are listed (Section \ref{DASOS})
		\item and by the end everything is summed up (Section \ref{DASOS-Vol-Summary}).
	\end{itemize}
	
	
	\section{State-of-Art FW LiDAR Software Packages}\label{LiDARsoftwares}
	
	\par The most common approach of interpreting the FW LiDAR is the Gaussian decomposition of the waveforms and extraction of peak points ~\cite{Wanger2006}. Neunschwander et al used this approach for Landcover classification ~\cite{Neuenschwander2009} while Reitberger et al applied it for distinguishing deciduous trees from coniferous trees ~\cite{Reitberger2008}. Chauve et al further proposed an approach of improving the Gaussian model in order to increase the density of the points extracted from the data and consequently improve point based classifications of FW LiDAR data ~\cite{Chauve2007}.  The following tools are able to extract discrete points from the waveforms and visualise small areas of interest:
	 
	\begin{itemize}
		\item \textbf{Pulsewaves}: it visualise a small number of waveform using different transparencies according to the intensities of the wave-samples and it is able to generate discrete point clouds.  \cite{Isenburg2012Pulsewaves}.
	    \item \textbf{FullAnalyze}: supports echo decomposition. Regarding visualisations the user can select single trees from the Graphical User Interface (GUI) and for each wave-sample, a sphere with radius proportional to its amplitude is created and visualised \cite{Chauve2009}. 
    	\item \textbf{SPDlib}: exports discrete LiDAR and visualises either the samples that are above a threshold level as points or the extracted discrete point cloud. It also colours them according to their intensity value\cite{Bunting2013}. 
	\end{itemize}

	\par  Echo decomposition and extraction of peak points identifies significant features and further enables the interpretation of the data within existing workflows and software that support discrete LiDAR data. A examples are: 
	
	\begin{itemize}
	\item \textbf{Lag}: a visualisation tool for analysing and inspecting discrete LiDAR point clouds. 
	
	\item \textbf{Quick Terrain Modeller} : a 3D discrete LiDAR points visualiser, that can generate Digital Elevation Models (DEM) and Digital Terrain Models (DTM).
	
	\item \textbf{LAStools} : a tool set that classifies noise, visualises point clouds, clips data.  
	\end{itemize}
		
	\par On the one hand, converting FW LiDAR into discrete, their usage is ease since existing overflows support discrete LiDAR, but on the other hand FW LiDAR contain information about pulse width that are not preserved after peak point extraction. Also the comparison of point clouds depends on the density of the emitted pulses; problem arises with the sinusoidal pattern of the Leica system. These information can be accumulated from multiple shots into a voxel array, building up a 3D density volume. The correlation between multiple pulses into a voxel representation produces a more accurate and complete representation, which confers greater noise resistance and it further opens up possibilities of vertical interpretation of the data. 
	
		
	 \section{Voxelisation for Interpreting FW LiDAR data}\label{Voxelisation}
		
	\par Voxelisation of FW LiDAR data was firstly introduced by Persson et al, who used it to visualise the waveforms using different transparencies \cite{Persson2005} and it seems to be the future of FW LiDAR data with the literature moving toward that direction. In 2016, Cao et al used it for tree species identification \cite{Cao2016} and Sunmall et al characterised forest canopy from a voxelised vertical profile \cite{Sumnall2016}. This innovative approach is an integral part of this thesis and it is used for both visualisations and classifications \cite{Miltiadou2014}\cite{Miltiadou2015}. 

		
		
	\par The FW LiDAR data are voxelised by inserting the wave samples into a 3D regular grid and constructing a 3D discrete density volume. According to Person et al, each wave sample is associated with the 3D cell, named voxel, that it lies inside. If multiple samples lie inside a voxel then the sample with the highest intensity is chosen ~\cite{Persson2005}. In order to reduce noise, there are two differences between this approach and the way FW LiDAR data are voxelised in DASOS. 
		
		
	\par At first a threshold is used to remove low level noise, because when the width of a recorded waveform is longer that the distance between the first hit point and the ground, the system captures low signals, which are pure noise. For that reason, the samples whose intensity is lower than a user-defined noise level/threshold are discarded. 
		
		
		\par Then each wave sample is associated with the voxel that it lies inside and the second difference is how DASOS overcomes the uneven number of samples per voxels. The intensity of each sample is the laser intensity returned during the corresponding time interval. For example, if 5 samples are inside a voxel and the waveform is digitised at 2ns, then the laser intensity associated with that voxel corresponds to 10ns waveform width. For comparison purposes, it's essential to keep the waveform width consistent across the voxels. For overcoming this issue in DASOS, the average intensity of the samples that lie inside each voxel is taken, instead of choosing the one with the highest intensity ~\cite{Persson2005}. This way the likelihood of the 3D volume to be affected by outliers and high noise is reduced. The following equation shows how the intensity value of a voxel is calculated:
		
		\begin{eqnarray}
		I_{v} = \dfrac{\sum_{i=1}^{n}I_{i}}{n}
		\end{eqnarray} 
		where 		$I_{v}$ is the accumulated intensity of voxel $v$. 
		$n$ is number of samples associated with that voxel, 
		$I_{i}$ is the intensity of the sample i, 
		
		To sum up, during voxelisation the area of interest is divided into voxels. The samples of the FW LiDAR data are inserted inside this 3D discrete density volume and normalised such that equally sized waveform width is saved inside each voxel. The result is a 3D discrete density volume of the scanned area. Figure \ref{fig:Voxelisation} depicts this process in 2D.
		
		
		\begin{figure} [h!]
			
			
			\begin{subfigure}[t]{.31\textwidth}
				
				\centering
				\includegraphics[width=.9\textwidth]{img/VoxelisationA}
				\caption{The sensor from the plane emits multiple pulses and collects information from the returned laser intensity.}
				\label{fig:VoxelisationA_scan}
			\end{subfigure} \hfill
			\begin{subfigure}[t]{.31\textwidth}
				\centering
				\includegraphics[width=.9\textwidth]{img/VoxelisationB}
				\caption{The area of interest is divided into equally sized cubes, named voxels, generating this way a discrete volume.} 
				\label{fig:VoxelisationB_grid}
			\end{subfigure} \hfill
			\begin{subfigure}[t]{.31\textwidth}
				\centering
				\includegraphics[width=.9\textwidth]{img/VoxelisationC}
				\caption{The accumulated intensities of wave samples into the volume build up the voxelised representation of the scanned area.} 
				\label{fig:VoxelisationC_voxelised}
			\end{subfigure}
			\caption[Voxelisation of FW LiDAR data]{The above images depict the voxelisation process of the FW LiDAR data in 2D. Please note that the voxelisation output in Figure \ref{fig:VoxelisationC_voxelised} shows how ideally the result would look. But in reality, a number of trees may be disconnected from the ground due to missing information about their trunk.}
			\label{fig:Voxelisation}
		\end{figure}
		
		
	
		
		
	\section{The functionalities of DASOS}\label{DASOS}
	
	   	\par DASOS is the open source software that was developed along with this thesis and its aim to ease the handling of FW LiDAR data. Table \ref{DASOS-functionalities} explains the three main the functionalities of DASOS, while the rest of the thesis explains the algorithms implemented and some of the related applications.      
        	
     
        		
	    \begin{longtable}[h!]
        {| p{0.08\linewidth}|p{0.3\linewidth}  | p{0.4\linewidth} | p{0.1\linewidth}|  }
        \toprule
        \multicolumn{4}{|c|}{\textbf{1st Functionality: 3D polygon Mesh Generation }} \\
        \toprule
        \multicolumn{4}{|c|}{Section \ref{Visualisations}} \\
        \toprule
        Input&Description & Example & Format \\ 
  		\cmidrule(r){1-1}\cmidrule(lr){2-2}\cmidrule(lr){3-3}\cmidrule(l){4-4}
   		LAS1.3& \textbf{ 3D polygon mesh } \newline & \raisebox{-\totalheight}{\adjincludegraphics[width=\linewidth,trim={{.1\width} 0 {.25\width} 0},clip]{img/NewForest}} & .asc \\ 
   		
        \bottomrule
                	\multicolumn{4}{c}{} \\
        
        	\toprule
        	\multicolumn{4}{|c|}{\textbf{2nd Functionality: Generation of 2D metrics}} \\
        	\multicolumn{4}{|c|}{\textbf{aligned with hyperspectral imagery }} \\
        	\toprule
        	\multicolumn{4}{|c|}{Section \ref{Alignment}: a selection of the following metrics has been used} \\
        	\multicolumn{4}{|c|}: for generating tree coverage maps} \\
        	\toprule
        	\textbf{Input}&\textbf{Metric Description} \newline (L for LiDAR metrics \& H for hyperspectral metrics) & \textbf{Example} & \textbf{Output Format} \\ 
         	
    
         	
      		\cmidrule(r){1-1}\cmidrule(lr){2-2}\cmidrule(lr){3-3}\cmidrule(l){4-4}
       		LAS1.3& \textbf{L0 - Height: } \newline the distance between the top non-empty voxel and the lower boundaries of the volume.& \raisebox{-\totalheight}{\adjincludegraphics[width=\linewidth,trim={{.1\width} 0 {.25\width} 0},clip]{img/metrics/HEIGHT}} & .asc \\ 
       		
           	\cmidrule(r){1-1}\cmidrule(lr){2-2}\cmidrule(lr){3-3}\cmidrule(l){4-4}
           	LAS1.3& \textbf{L1 - Thickness: } \newline the distance between  the  first  and  last  non  empty  voxels  in every   column of the   3D   volume. .& \raisebox{-\totalheight}{\adjincludegraphics[width=\linewidth,trim={{.1\width} 0 {.25\width} 0},clip]{img/metrics/THICKNESS}} & .asc \\ 
         	
			\cmidrule(r){1-1}\cmidrule(lr){2-2}\cmidrule(lr){3-3}\cmidrule(l){4-4}
			LAS1.3&   \textbf{L2 - Edge detection:}\newline The average height difference of neighbouring pixels. &         	\raisebox{-\totalheight}{\adjincludegraphics[width=\linewidth,trim={{.1\width} 0 {.25\width} 0},clip]{img/metrics/AverageHeightDiff}} & .asc \\ 
			
        	\cmidrule(r){1-1}\cmidrule(lr){2-2}\cmidrule(lr){3-3}\cmidrule(l){4-4}				
        	LAS1.3& \textbf{L3 - Density:} \newline Number of non-empty voxel over all voxelswithin  the  range  from the first to last non-empty voxels. &       \raisebox{-\totalheight}{\adjincludegraphics[width=\linewidth,trim={{.1\width} 0 {.25\width} 0},clip]{img/metrics/DENSITY}} & .asc \\ 
        	
        	\cmidrule(r){1-1}\cmidrule(lr){2-2}\cmidrule(lr){3-3}\cmidrule(l){4-4}
        	LAS1.3& \textbf{L4 - First Patch: } \newline The number of non-empty adjacent voxels, starting from the first/top non-empty voxel in that column. &         					\raisebox{-\totalheight}{\adjincludegraphics[width=\linewidth,trim={{.1\width} 0 {.25\width} 0},clip]{img/metrics/FIRST_PATCH}} & .asc \\ 
        	
        	\cmidrule(r){1-1}\cmidrule(lr){2-2}\cmidrule(lr){3-3}\cmidrule(l){4-4}
        	LAS1.3& \textbf{L5: Last Patch: } \newline The number of non-empty adjacent voxels, starting from   the last/lower   non-empty   voxelin   that column.& \raisebox{-\totalheight}{\adjincludegraphics[width=\linewidth,trim={{.1\width} 0 {.25\width} 0},clip]{img/metrics/LAST_PATCH}} & .asc \\
        	
        		\cmidrule(r){1-1}\cmidrule(lr){2-2}\cmidrule(lr){3-3}\cmidrule(l){4-4}
        		LAS1.3& \textbf{L6: Lowest Return: } \newline & \raisebox{-\totalheight}{\adjincludegraphics[width=\linewidth,trim={{.1\width} 0 {.25\width} 0},clip]{img/metrics/LOWEST_RETURN}} & .asc \\
        		
        			\cmidrule(r){1-1}\cmidrule(lr){2-2}\cmidrule(lr){3-3}\cmidrule(l){4-4}
        			LAS1.3& \textbf{L7: Maximum Intensity: } \newline & \raisebox{-\totalheight}{\adjincludegraphics[width=\linewidth,trim={{.1\width} 0 {.25\width} 0},clip]{img/metrics/metric_INTENSITY_MAX}} & .asc \\
        			
        				\cmidrule(r){1-1}\cmidrule(lr){2-2}\cmidrule(lr){3-3}\cmidrule(l){4-4}
        				LAS1.3& \textbf{L8: Average Intensity: } \newline & \raisebox{-\totalheight}{\adjincludegraphics[width=\linewidth,trim={{.1\width} 0 {.25\width} 0},clip]{img/metrics/metric_INTENSITY_AVG}} & .asc \\
        	  	   	
        	
        	\cmidrule(r){1-1}\cmidrule(lr){2-2}\cmidrule(lr){3-3}\cmidrule(l){4-4}
        	LAS1.3 \newline and \newline level 1 (.bil \& .igm)& \textbf{H0 : Mean } \newline the mean of the hyperspectral spectrum.& \raisebox{-\totalheight}{\adjincludegraphics[width=\linewidth,trim={{.1\width} 0 {.25\width} 0},clip]{img/metrics/HYPERSPECTRAL_MEAN}} & .asc \\ 
        
     	   	\cmidrule(r){1-1}\cmidrule(lr){2-2}\cmidrule(lr){3-3}\cmidrule(l){4-4}
     	   	LAS1.3 \newline and \newline level 1 (.bil \& .igm&\textbf{H1: NDVI } \newline Normalised Difference Vegetation index.& \raisebox{-\totalheight}{\adjincludegraphics[width=\linewidth,trim={{.1\width} 0 {.25\width} 0},clip]{img/metrics/NDVI}} & .asc \\ 
     	   	
     	   	\cmidrule(r){1-1}\cmidrule(lr){2-2}\cmidrule(lr){3-3}\cmidrule(l){4-4}
     	   	LAS1.3 \newline and \newline level 1 (.bil \& .igm&\textbf{H2: Standard \newline Deviation *} \newline The standard deviation of the hyperspectral spectrum at each pixel.& \raisebox{-\totalheight}{\adjincludegraphics[width=\linewidth,trim={{.1\width} 0 {.25\width} 0},clip]{img/metrics/std}} & .asc \\ 
     	   	
     	   	\cmidrule(r){1-1}\cmidrule(lr){2-2}\cmidrule(lr){3-3}\cmidrule(l){4-4}
     	   	LAS1.3 \newline and \newline level 1 (.bil \& .igm&\textbf{H3: Spectral \newline Signature *} \newline The   squared spectral   difference   between   each pixels’  spectrum  and the  generalised vegetation signature retrieved  from  USGS  Digital  Spectral Library \cite{Clark2007}.& \raisebox{-\totalheight}{\adjincludegraphics[width=\linewidth,trim={{.1\width} 0 {.25\width} 0},clip]{img/metrics/spectralSignature}} & .asc \\ 
     	   	   	
     	   	\cmidrule(r){1-1}\cmidrule(lr){2-2}\cmidrule(lr){3-3}\cmidrule(l){4-4}
     	   	LAS1.3 \newline and \newline level 1 (.bil \& .igm&\textbf{H4: Band} \newline A single user defined hyperspectral band. & \raisebox{-\totalheight}{\adjincludegraphics[width=\linewidth,trim={{.1\width} 0 {.25\width} 0},clip]{img/metrics/NDVI}} & .asc \\ 
     	   		
     	   		
     	   	\bottomrule
        	
        	\multicolumn{4}{c}{} \\
      \toprule
      \multicolumn{4}{|c|}{\textbf{3rd Functionality: 3D Priors / Signatures }} \\
      \toprule
      \multicolumn{4}{|c|}{Section \ref{Classifications}} \\
      \toprule
      Input&Description & Example & Format \\ 
      \bottomrule
        					
        	

        	\caption{my.Lboro Analysis}
        	\label{tbl:myLboro}

        \end{longtable}
        		

	   	
	   		\subsection{2D Metrics Aligned with Hyperspectral Imagery}
	   		
	   		- Generation of 2D metrics aligned with hyperspectral imagery 
	   		
	   		\subsection{Polygon Meshes Generation}
	   		- Exports 3D coloured polygons meshes into a standard graphics format
	   		
	   		\subsection{3D Prior and Signatures}
	   		- Generats 3D priors and saves them into .csv easy to use in R and matlab for classifications. 
	   		
	   		\subsection{DASOS's Users and Future Directions}
	   		
	   		can produce 3D polygons and metrics useful in deriving information about the scanned areas
	   		
	   		
	   	
	   	\par Please note that all the information about DASOS are given in this Blogpost \url{<http://miltomiltiadou.blogspot.co.uk/2015/03/las13vis.html>}, which also contains the up-to-date links for downloading the user-guide or source code, as well as the link to the related Google Group for users support. 
	   	 
       
	   	
	   
	   	
	  \section{Summary and Discussion} \label{DASOS-Vol-Summary}
	  
	   	Along with that thesis, the open source software DASOS was developed to encourage foresters use the FW LiDAR data. The main way of interpreting FW LiDAR data in DASOS is fundamentally different from the state-of-art available software managing FW LiDAR Data. In a few words, the FW LiDAR data are voxelised by inserting the wave samples into a 3D discrete density volume, which preserves an extra parameter (the pulse width) in comparison to point extraction algorithms. It also accumulates intensity values from multiple shots and stores them into a 3D regular grid, resolving this way the problem with the sinusoidal footprints pattern of the Leica system.
	   	
	   	Furthermore, there are three main functionalities of DASOS: the generation of 2D metrics aligned with hyperspectral Images, the construction of 3D polygon meshes and characterisation of objects using 3D priors/signatures. Integration of various sensors of data allows simultaneous interpretation of them and in Section \ref{Alignment}, it is shown that this confers better results. In addition, the visualisation outputs are also in the cutting-edge since previous visualisation talk about points \cite{Bunting2013} or spheres \cite{Chauve2009}, while DASOS is able to create solid coloured polygon representation. ***. The algorithms and applications of those functionalities are explained in Sections \ref{Alignment}, \ref{Visualisations}, \ref{Classifications} respectively. 
	   	

	   		
	   	So far, there a few individuals/organisation that showed interest in using DASOS, but in the future, DASOS usage is expected to increased in remote forest surveys (i.e. for commercial forest’s stocking estimation or for infected trees detection and treatment).
	   	
	   	
\end{document}