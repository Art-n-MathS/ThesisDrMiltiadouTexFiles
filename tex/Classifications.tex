\documentclass{subfiles}

\begin{document}


This talk presents the new features of DASOS, which is an open source software for managing full-waveform LiDAR data and those features are used for detecting dead standing Eucalypt.

The value of dead standing Eucalypt trees from a biodiversity management perspective is large. In Australia, many arboreal mammals and birds that are close to extinct inhabit hollows ~\cite{Gibbons2002}. Nevertheless, studies predict shortage of hollows in the near future due to tree harvesting and the decades required for a tree to be mature enough to develop a hollow ~\cite{Lindenmayer2010} ~\cite{Goldingay2009}. Dead standing eucalypt trees are more likely to be aged and have hollows, therefore automated detection of them plays a significant role in protecting animals that rely on hollows. 

DASOS (= $\delta\acute{a}\sigma o\varsigma$) means forest in Greek and it is an open source software aiming to ease the way of handling FW LiDAR data in forestry ~\cite{Miltiadou2015}. Traditional ways of interpreting FW LiDAR data, suggests extraction of a denser points cloud using Gaussian decomposition ~\cite{Neuenschwander2009} ~\cite{Reitberger2008}. Nevertheless DASOS was influenced by Persson et al, 2005, who used voxelisation to visualise the waveforms ~\cite{Persson2005}. But, DASOS do not only uses voxelisation for visualisations but also for extracting metrics useful in classification. It further normalises the intensities so that equal pulse length exists inside each voxel, making intensities more meaningful. It is further seems that the literature is moving towards voxelisation with the good results obtained at recent publication on tree species classification ~\cite{Cao2016}. 

The new features of DASOS: New features of DASOS which enables observation at tree level: i.e. distribution of intensities at specific area

The data, provided by RPS Australia East Pty Ltd, were collected in March 2015 from the Riegl ( LMS-Q780 or LMS-Q680i?) sensor at an Australian native Forest with eucalyptus. The fieldplots has been provided by (Interprine Group Ltd or Forest Corporation?). 

examined with Random Forest



The new features of DASOS are presented and used for generating 3D signatures characterising dead standing trees and a comparison between the discrete and FW LiDAR data is performed to demonstrate the increased survey accuracy obtained with the FW LiDAR. 






\end{document}