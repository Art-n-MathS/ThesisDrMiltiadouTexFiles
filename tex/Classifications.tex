\documentclass{subfiles}

\begin{document}


This talk presents the new features of DASOS, which is an open source software for managing full-waveform LiDAR data and those features are used for detecting dead standing Eucalypt.

The value of dead standing Eucalypt trees from a biodiversity management perspective is large. In Australia, many arboreal mammals and birds that are close to extinct inhabit hollows ~\cite{Gibbons2002}. Nevertheless, studies predict shortage of hollows in the near future due to tree harvesting and the decades required for a tree to be mature enough to develop a hollow ~\cite{Lindenmayer2010} ~\cite{Goldingay2009}. Dead standing eucalypt trees are more likely to be aged and have hollows, therefore automated detection of them plays a significant role in protecting animals that rely on hollows. 

DASOS (= $\delta\acute{a}\sigma o\varsigma$) means forest in Greek and it is an open source software aiming to ease the way of handling FW LiDAR data in forestry ~\cite{Miltiadou2015}. Traditional ways of interpreting FW LiDAR data, suggests extraction of a denser points cloud using Gaussian decomposition ~\cite{Neuenschwander2009} ~\cite{Reitberger2008}. Nevertheless DASOS was influenced by Persson et al, 2005, who used voxelisation to visualise the waveforms ~\cite{Persson2005}. But, DASOS do not only uses voxelisation for visualisations but also for extracting metrics useful in classification. It further normalises the intensities so that equal pulse length exists inside each voxel, making intensities more meaningful. It is further seems that the literature is moving towards voxelisation with the good results obtained at recent publication on tree species classification ~\cite{Cao2016}. 

The new features of DASOS: New features of DASOS which enables observation at tree level: i.e. distribution of intensities at specific area

The data, provided by RPS Australia East Pty Ltd, were collected in March 2015 from the Riegl ( LMS-Q780 or LMS-Q680i?) sensor at an Australian native Forest with eucalyptus. The fieldplots has been provided by (Interprine Group Ltd or Forest Corporation?). 

examined with Random Forest



The new features of DASOS are presented and used for generating 3D signatures characterising dead standing trees and a comparison between the discrete and FW LiDAR data is performed to demonstrate the increased survey accuracy obtained with the FW LiDAR. 


This paper presents a new feature of DASOS, which is an open source software for managing full-waveform (FW) LiDAR data and that feature is used for detecting dead standing Eucalypt trees in native Australian forests.

The value of dead standing Eucalypt trees from a biodiversity management perspective is large. In Australia, many arboreal mammals and birds, which are close to extinct, inhabit hollows ~\cite{Gibbons2002}. Nevertheless, studies predict shortage of hollows in the near future due to tree harvesting and the decades required for a tree to develop a hollow ~\cite{Lindenmayer2010}~\cite{Goldingay2009}. Dead standing eucalypt trees are more likely to be aged and have hollows, therefore automated detection of them plays a significant role in protecting animals that rely on hollows. 

The LiDAR data used for this project are provided by RPS Australia East Pty Ltd and they were collected in March 2015 using the Riegl ( LMS-Q780 or LMS-Q680i?) sensor. The Riegl LMS-Q??? is a native full-waveform sensor and the LiDAR point clouds were generated from the waveform instrument data during post processing. In addition, the field plots used for the classifications are provided by (Interprine Group Ltd or Forest Corporation?) and contain around 1000 Eucalypt trees while 10\verb|%| of them are dead. 

The new feature of DASOS calculates forestry metrics within a radius relevant to canopy height and exports all metrics into a single vector for fast interpretation in advanced statistical tools.  Traditional ways of interpreting FW LiDAR data, suggests extraction of a denser points cloud ~\cite{Neuenschwander2009}~\cite{Reitberger2008}, but as mentioned before with the Riegl system this is done at post processing. Nevertheless DASOS was influenced by Persson et al, 2005, who used voxelisation to visualise the waveforms ~\cite{Persson2005}, but DASOS also uses it for generating metrics. It further normalises the intensities so that equal pulse length exists inside each voxel, making intensities more meaningful. Further, recent publication on tree species classification showed that voxelisation could confer good results while interpreting FW LiDAR data ~\cite{Cao2016}. 

Previous work on dead standing trees detection, suggests single tree segmentation before dead trees identification ~\cite{Yao2012}~\cite{Polewski2015} but in case of Eucalypt trees single tree detection is a challenge on its own due to their irregular structure and multiple trunk splits. 

In this project, the new feature of DASOS is used for generating 3D signatures characterising dead standing Eucalypt trees and a comparison between the LiDAR point cloud and FW LiDAR data is performed using Random Forest to demonstrate the increased survey accuracy obtained with voxelisation.





\end{document}