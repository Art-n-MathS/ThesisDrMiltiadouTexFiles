\documentclass{subfiles}

\begin{document}

\section{Introduction}



\section{The Importance of Dead Wood in our Ecosystem}

\par The value of dead trees from a biodiversity management perspective is large. Once a tree dies, its contribution to our ecosystem continues. The woody structure remains for centuries and it contributes to forest regeneration while providing resources for numerous surrounding organisms \cite{Franklin1987}. As an indication, more than 4000 species inhabit dead wood in Finland \cite{Siitonen2001}, where an estimate of 1000 species is extinct \cite{Hanski2000}. These species do not only include animals and birds but also organisms, like fungi. Fungi contributes to wood decaying and biodiversity, which is an important factor for a resilient ecosystem \cite{Peterson2000}. Observing the changes of fungal diversity on decaying wood has an increased interest in science  \cite{Abrego2011} \cite{Stokland2011} \cite{Lonsdale2008} in order to ensure the continuous existence of decaying wood in forests. 




\par In Australia, many arboreal mammals and birds that are close to extinct inhabit hollows \cite{Gibbons2002}. Nevertheless, studies predict shortage of hollows in the near future due to tree harvesting and the decades required for a tree to be mature enough to develop a hollow \cite{Lindenmayer2010} \cite{Goldingay2009}. Dead standing eucalypt trees are more likely to be aged and have hollows, therefore automated detection of them plays a significant role in protecting animals that rely on hollows. 





\section{Related Work}

Dead trees could either be fallen or standing. The task of identifying them from a classification perspective is different. Fallen trees ***  \cite{Polewski2015} \cite{Mucke2013}
while standing trees ***\cite{Yao2012}. This research is focused on standing dead tree detection and to be more specific in native Eucalypt forest in Australia. 


\par Previous work on dead standing trees detection, suggests single tree segmentation before dead trees identification \cite{Yao2012} but in case of Eucalyptus single tree detection is a challenge on its own due to their irregular structure and multiple trunk splits. 

Research on top tree detection with maxima and watershed for delineation. 

An interesting work for delineating Eucalyptus recommends delineation from bottom to top. It first identifies standing trunks and then segments the points cloud. 

Tree delineation from bottom to top \cite{Shendryk2016_treeDeliniation}


This is then used for monitoring forest health \cite{Shendryk2016_DeadTrees}

Nevertheless, this approach is time consuming. We introduced a fast approach for rough classification similar to boost cascade \cite{Viola2001} but extended to 3D. 




\par Traditional ways of interpreting FW LiDAR data, suggests extraction of a denser points cloud using Gaussian decomposition ~\cite{Neuenschwander2009} ~\cite{Reitberger2008}. Nevertheless DASOS was influenced by Persson et al, 2005, who used voxelisation to visualise the waveforms ~\cite{Persson2005}. But, DASOS do not only uses voxelisation for visualisations but also for extracting metrics useful in classification. It further normalises the intensities so that equal pulse length exists inside each voxel, making intensities more meaningful. It is further seems that the literature is moving towards voxelisation with the good results obtained at recent publication on tree species classification ~\cite{Cao2016}. 

\section{Data and Fieldplots}

\par The data, provided by RPS Australia East Pty Ltd, were collected in March 2015 from the Riegl ( LMS-Q780 or LMS-Q680i?) sensor at an Australian native Forest with eucalyptus. The fieldplots has been provided by (Interprine Group Ltd or Forest Corporation?). 
The LiDAR data used for this project are provided by RPS Australia East Pty Ltd and they were collected in March 2015 using the Riegl ( LMS-Q780 or LMS-Q680i?) sensor. The Riegl LMS-Q??? is a native full-waveform sensor and the LiDAR point clouds were generated from the waveform instrument data during post processing. In addition, the field plots used for the classifications are provided by (Interprine Group Ltd or Forest Corporation?) and contain around 1000 Eucalypt trees while 10\verb|%| of them are dead. 

\section{Methodology}

\par In this chapter, the 3rd feature of DASOS (Table \ref{tbl:functionalities}) is used for generating 3D priors characterising dead standing Eucalypt trees. These 3D priors are used for detecting dead standing Eucalypt trees in native Australian forests. 

\section{Experiments and Results}

\section{Conclusions}













\end{document}