
\documentclass{subfiles}

\begin{document}
	

	
\par This research aims to ease the way of handling full-waveform (FW) LiDAR  data for remote forest surveys. 
\par The primary output of this thesis is the open source software DASOS (=forest in Greek), which aims to break the barrier between understanding and using the FW LiDAR. New representations of the FW LiDAR are further proposed for handling the data and by using those representations, DASOS can produce 3D polygons and metrics useful in deriving information about the scanned areas. The contributions of the DASOS and the new representations of the FW LiDAR are demonstrated in three applications: alignment with hyperspectral imagery, dead tree detection and optimised volumetric visualisation.

\end{document}



Some animals
have important roles in ecosystem processes and organization,
such as pollination, seed dispersal, and
herbivory, and the loss of these species has clear
negative consequences for ecosystem resilience (e.g.,
Elmqvist et al. 2003).