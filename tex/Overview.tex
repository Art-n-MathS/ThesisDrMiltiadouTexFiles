
\documentclass{subfiles}

\begin{document}
\par This research aims to ease the way of handling full-waveform (FW) LiDAR  data for remote forest surveys. Automated ways of monitoring forests are required to reduce the time consuming and expensive fieldwork undertaken for monitoring both commercial and sustainable forests. The discrete LiDAR has been widely used and a 40\verb|%| reduction of fieldwork has been achieved at Interpine Ltd Group, New Zealand, with that technology. FW LiDAR systems are novel in that they record the entire backscatter signal of the emitted pulse, rather than a few returns common with conventional discrete LiDAR. This suggests many new problems and possibilities from the point of view of image understanding, remote surveying and visualisation. Scientists understand their concepts and potentials but due to the lack of available tools able to handle these large datasets, there are very few uses of FW LiDAR (Anderson et al, 2015).

\par The primary output of this thesis is the open source software DASOS (=forest in Greek), which aims to break the barrier between understanding and using the FW LiDAR. New representations of the FW LiDAR are further proposed for handling the data and by using those representations, DASOS can produce 3D polygons and metrics useful in deriving information about the scanned areas. The contributions of the DASOS and the new representations of the FW LiDAR are demonstrated in three applications: alignment with hyperspectral imagery, dead tree detection and optimised volumetric visualisation.

\end{document}



