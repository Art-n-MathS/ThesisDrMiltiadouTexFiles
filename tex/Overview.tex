
\documentclass{subfiles}

\begin{document}
	

	
\par This research aims to ease the way of handling full-waveform (FW) LiDAR  data for remote forest surveys. The primary output of this thesis is the open source software DASOS (=forest in Greek), which aims to break the barrier between understanding and using the FW LiDAR. New representations of the FW LiDAR are further proposed for handling the data and by using those representations, DASOS can produce 3D polygons and metrics useful in deriving information about the scanned areas. The contributions of DASOS and the new representations of the FW LiDAR are demonstrated in three applications: 

\begin{itemize}
	\item Visualisations and optimisations on managing real volumetric data:
	
	\item Alignment with hyperspectral images for generating tree coverage maps:
	
	\item Dead Tree detection for managing biodiversity in Australian native forests:
	

\end{itemize}

\begin{figure}
	\includegraphics[width=\textwidth]{tex/Pipeline/Pipeline.png}
	\caption{The pipeline of the thesis}
\end{figure}



\end{document}


