
\documentclass{subfiles}

\begin{document}

\section{Problem}\label{sec:Problem}
\par FW LiDAR systems have been available for a number of years but there still very few uses of FW LiDAR data. NERC-ARF has been acquiring airborne data for the UK and overseas since 2010 and it has more than 100 clients of new and archived data. Many clients request FW LiDAR data to be acquired, but despite the significant number of requests,the majority of research still only uses discrete LIDAR. Some of the factors regarding this slow intakes are:
\begin{itemize}
	\item Typically FW datasets are 5 – 10 times larger than discrete data, with data sizes in the range of 50GB – 2.5TB GB for a single area of interest. NERC-ARF's datasets are up to 100GB each because most clients are research institutes but for commercial purposes each FW dataset is a couple of TB.
	\item Existing workflows are only able to work with the discrete data since the increased amount of information recorded within the FW LiDAR makes handling the quantity of data very challenging.
\end{itemize}

\newpage

\section {Aims and Objectives}\label{sec:AimsObjectives}

\par This thesis explores visualisation and data-understanding for FW LIDAR systems and the overarching aim is to increase the accessibility FW LiDAR in remote forest surveying. {\color{blue}The objectives are listed in Table \ref{tab:Objectives} and they are associated with the Sections that tackles them.}

\begin{table}[!htbp]
	% increase table row spacing, adjust to taste
	\renewcommand{\arraystretch}{1.3}
	% if using array.sty, it might be a good idea to tweak the value of
	% \extrarowheight as needed to properly center the text within the cells
	
	\centering
	% Some packages, such as MDW tools, offer better commands for making tables
	% than the plain LaTeX2e tabular which is used here.
	\begin{tabular}{|P{.1\textwidth}|P{.7\textwidth}|P{.2\textwidth}|}	
		\hline
		\textbf{No.} &	\textbf{Objective} &	\textbf{Related Chapters}  \\
		\hlinewd{1.5pt}
		1 &	Enable forestry experts with no computer science expertise to visualise and work with the FW LiDAR data.  &	\ref{Visualisations} \\	
		\hline
		2 &	Enable forest understanding through 3D visualisations of FW LiDAR data. &	\ref{Visualisations}  \\
		\hline
		3 &	Improve and optimise visualisations of FW LiDAR data and hyperspectral images. &	\ref{Optimisations} \& \ref{Alignment}  \\
		\hline
		4 &	Enable browsing of very large scale datasets and spectral bands in an efficient manner. &	\ref{Optimisations} \& \ref{Alignment}   \\
		\hline
		5 &	Investigate data structures for faster iso-surface extraction of large volumetric datasets and efficient management of voxels. &	\ref{Optimisations} \\	
		\hline
		6 &Estimate tree coverage and investigate the potential of integrating multiple remote sensing datasets in forestry. &	\ref{Alignment}  \\
		\hline
		7 &	Dead tree detection in comparison to human detection and remote surveying with discrete LiDAR that will benefits biodiversity management. &	\ref{Classifications}  \\
		\hline
		8 &	Research whether terrain classification can be improved by the inference of high quality 3D information, for example, using priors over the space of 3D elements. &	\ref{Classifications}  \\
		\hline
		
		\hline
	\end{tabular}
	\caption{Values of divisible sides}
	\label{tab:Objectives}
\end{table}
\newpage
\section{Overview}

\par {\color{red} *** the following text has been taken from the IAA2 funding application}
	
\par To address the limitations of existing workflows for using with FW data we developed the open source software DASOS ( from $\delta \alpha \sigma o \varsigma$ meaning forest in Greek) and novel algorithms that allow users, without computer science expertise, to work with and visualise large volumes of FW LiDAR data. Our open source software DASOS aims to remove the barriers preventing the use of FW LiDAR. Its contributions, and those of the new representations of the FW LiDAR, are demonstrated in three applications:

\begin{itemize}
\item Firstly, foresters can exploit their domain expertise to derive a wealth of information by observing the FW LiDAR data. We therefore improve visualisations for deriving information directly from the data, thus reducing travelling time and the associated expenses of getting into the forests. This cost includes appropriate cars and sometimes helicopters depending on the accessibility of the forests. While previous work on FW LiDAR visualisation talks about point cloud visualisation \cite{Isenburg2012Pulsewaves} and transparent voxels \cite{Persson2005}, DASOS is able to reconstruct the surfaces from the scanned area in 3D. This research further optimisesvisualisations by using the new FW LiDAR representations to accelerate this process by ****\%. {\color{red} *** I will complete the percentage once related test are completed}

\item Secondly, a fast way of aligning the FW LiDAR with Remotely Sensed Images has been developed in DASOS. Subsequently, by generating tree coverage maps, it has been shown that the combination of these datasets confers better remote survey results \cite{Miltiadou2015}.


\item {\color{gray} Finally, DASOS allows the generation of 3D priors. An example usage of this information is characterising dead standing Eucalyptuses, which as explained at Section \ref{sec:forestMonitoring} are extremely beneficial for managing biodiversity in native Australian forests. This is work in progress and a comparison between the discrete and FW LiDAR will be performed to demonstrate the increased survey accuracy obtained when the FW LiDAR is used.}

\end{itemize}



\par In summary, FW LiDAR has great potential to improving automated surveying accuracy and consequently reduce the expensive fieldwork conducted in forestry and this research has already started to have an impact in the FW LiDAR community. DASOS is now used at Interpine Group Ltd, a world leading Forestry Company in New Zealand, and a PhD student at Bournemouth University is evaluating it for use in the estimation of bird distributions in the New Forest in the UK.

	
	\par {\color{red} *** end of copied text}
	


\section{Thesis Structure}

\begin{figure}[!htbp]
	\includegraphics[width=\textwidth]{img/Pipeline}
	\caption{The pipeline of the thesis}
\end{figure}





\end{document}


