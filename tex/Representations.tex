\documentclass{subfiles}


\begin{document}

\section{Introduction}
	\par {\color{red} *** Neill: the following paragraphs were moved here from the Background Chapter 1.2}
		



	\par While echo decomposition identifies significant features as points, the FW LiDAR data also contains information in single shots that may be below the significance threshold. The waveform samples data can be accumulated from multiple shots into {\color{blue} into a voxel array, building up a 3D density volume. The correlation between multiple pulses into a voxel representation produces a more accurate and complete representation, which confers greater noise resistance and it further opens up possibilities of vertical interpretation of the data. Voxelisation (explained at Section \ref{sec:Vox:Approach}) of FW LiDAR data} was firstly introduced by Persson et al, who used it to visualise the waveforms using different transparencies \cite{Persson2005} and it seems to be the future of FW LiDAR data with the literature moving toward that direction. In 2016, Cao et al used it for tree species identification \cite{Cao2016} and Sunmall et al characterised forest canopy from a voxelised vertical profile \cite{Sumnall2016}. This innovative approach is an integral part of this thesis and it is used for both visualisations and classifications \cite{Miltiadou2014}\cite{Miltiadou2015}. 

	\par {\color{red} *** end of moved paragraphs *** }
		
		

	
\section{Approach}\label{sec:Vox:Approach}

\par The FW LiDAR data are voxelised by inserting the wave samples into a 3D regular grid and constructing a 3D discrete density volume. According to Person et al, each wave sample is associated with the 3D cell, named voxel, that it lies inside. If multiple samples lie inside a voxel then the sample with the highest intensity is chosen ~\cite{Persson2005}. In order to reduce noise, there are two differences between this approach and the way FW LiDAR data are voxelised in DASOS. 


\par At first a threshold is used to remove low level noise, because when the width of a recorded waveform is longer that the distance between the first hit point and the ground, the system captures low signals, which are pure noise. For that reason, the samples whose intensity is lower than a user-defined noise level/threshold are discarded. 


\par Then each wave sample is associated with the voxel that it lies inside and the second difference is how DASOS overcomes the uneven number of samples per voxels. The intensity of each sample is the laser intensity returned during the corresponding time interval. For example, if 5 samples are inside a voxel and the waveform is digitised at 2ns, then the laser intensity associated with that voxel corresponds to 10ns waveform width. For comparison purposes, it's essential to keep the waveform width consistent across the voxels. For overcoming this issue in DASOS, the average intensity of the samples that lie inside each voxel is taken, instead of choosing the one with the highest intensity ~\cite{Persson2005}. This way the likelihood of the 3D volume to be affected by outliers and high noise is reduced. The following equation shows how the intensity value of a voxel is calculated:
 
	\begin{eqnarray}
	I_{v} = \dfrac{\sum_{i=1}^{n}I_{i}}{n}
	\end{eqnarray} 
	where 		$I_{v}$ is the accumulated intensity of voxel $v$. 
	$n$ is number of samples associated with that voxel, 
	$I_{i}$ is the intensity of the sample i, 

To sum up, during voxelisation the area of interest is divided into voxels. The samples of the FW LiDAR data are inserted inside this 3D discrete density volume and normalised such that equally sized waveform width is saved inside each voxel. The result is a 3D discrete density volume of the scanned area. Figure \ref{fig:Voxelisation} depicts this process in 2D.


\begin{figure} [h!]


\begin{subfigure}[t]{.31\textwidth}

\centering
\includegraphics[width=.9\textwidth]{img/VoxelisationA}
\caption{The sensor from the plane emits multiple pulses and collects information from the returned laser intensity.}
\label{fig:VoxelisationA_scan}
\end{subfigure} \hfill
\begin{subfigure}[t]{.31\textwidth}
\centering
\includegraphics[width=.9\textwidth]{img/VoxelisationB}
\caption{The area of interest is divided into equally sized cubes, named voxels, generating this way a discrete volume.} 
\label{fig:VoxelisationB_grid}
\end{subfigure} \hfill
\begin{subfigure}[t]{.31\textwidth}
\centering
\includegraphics[width=.9\textwidth]{img/VoxelisationC}
\caption{The accumulated intensities of wave samples into the volume build up the voxelised representation of the scanned area.} 
\label{fig:VoxelisationC_voxelised}
\end{subfigure}
\caption[Voxelisation of FW LiDAR data]{The above images depict the voxelisation process of the FW LiDAR data in 2D. Please note that the voxelisation output in Figure \ref{fig:VoxelisationC_voxelised} shows how ideally the result would look. But in reality, a number of trees may be disconnected from the ground due to missing information about their trunk.}
\label{fig:Voxelisation}
\end{figure}


\section{Summary}

In a few words, the FW LiDAR data are voxelised by inserting the wave samples into a 3D discrete density volume. This thesis strongly supports voxelisation of FW LiDAR data and it is used for both visualisations and classifications. Voxelisation is also supported by recent literature \cite{Cao2016} \cite{Sumnall2016} because it preserves an extra parameter in comparison to point extraction algorithms, the pulse width. It also accumulates intensity values from multiple shots and stores them into a 3D regular grid, resolving this way the problem with the sinusoidal footprints pattern of the Leica system. 



\end{document}