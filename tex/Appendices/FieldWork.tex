\documentclass{subfiles}

\begin{document}
	\section{Introduction}
	\par This is a case study contains field work for a non-professional perspective to better understand the challenges of working remotely with forests. Remotely sensed data contain a great amount of information. But in order to build a good system for identifying trees and materials, an in depth knowledge of them is required \cite{Smith2012}. For that reason, this case study was held where information about the New Forest were collected and a small validation dataset was created. The dataset created includes the tree species and approximate heights of the trees on two areas of interest. 
	

	\par Before getting to New Forest, two areas of interest were selected. These areas were selected according to the following criteria:
	\begin{itemize}
		\item There were LiDAR data of the selected area to be able to compare what we can see in real life with the scanned data
		\item Areas that had a variation of tree species was selected. This was done according to the (non-validated) results of a thesis of Bournemouth University that classified the tree species of New Forest \cite{Sumnall2013}. This helped into getting a boarder view of tree species. 
	\end{itemize}
	
	\par The following sections give a detailed description of the information gathered during the trip. This includes the species and height maps generated, the different types of landscapes found and the challenges faced. 
		
	
	   \section{Validation Data Collected}
	   \par The tree classes were initially defined by the provided Bournemouth thesis \cite{Sumnall2013}. A colour was chosen for each tree class and while been in New Forest the aim was to mark each tree on the paper map with the corresponding colour. Using QGIS (Quantum Geographic Information System) the classification results of the forest assessment, held by Sumnall in 2013 \cite{Sumnall2013}, were coloured with the same colours to ease comparison.
	   
	   \par At the aforementioned forest assessment there were used 26 classes from 14 different species; the rest 12 classes were young versions of these 14 species. Here the classes are reduced to 14 by merging all the young trees to the normal tree species classes due to the 4 years gap that exists between the date the data were collected and the visit to new Forest (see table \ref{tab:TBL_InitialColours} for the initial 14 classes). Nevertheless, more tree species existed in the areas of interest in New Forest than those 14 classes. The colours and symbols of the extra tree classes are shown on table \ref{tab:TBL_AddedColours}. 
	   
	   \begin{table}[!h]
	   	\centering
	   	\begin{tabular}{c}
	   		\raisebox{-\totalheight}{\adjincludegraphics[width=\linewidth]{img/NewForest/TBLInitialColours}}
	   	\end{tabular}
	   	\caption{Colours of the initial 14 classes}
	   	\label{tab:TBL_InitialColours}
	   \end{table}
	   
	   \begin{table}[!h]
	   	\centering
	   	\begin{tabular}{c}
	   		\raisebox{-\totalheight}{\adjincludegraphics[width=\linewidth]{img/NewForest/TBLAddedColours}}
	   	\end{tabular}
	   	\caption{Classes that were added during the trip}
	   	\label{tab:TBL_AddedColours}
	   \end{table}
	   
	   \par During the visit, tree species maps were generated for a few square meters. The position of the trees were found relatively to well-spotted points (e.g. road crossing) that were marked in advanced. That was done because according to Dr. Ross Hill, no GPS can be accurate enough when trees are around since the signal bounces on the leaves and it captures faulty signals. In professional fieldwork a total station is used but the purpose of that visit it was not considered necessary. By the end of the case study, ground maps were coloured according to the tree species identified and the heights of the trees were also approximately noted down. 
	   
	   \par The following four maps was documented for each selected area. The first two maps were created before the trip during preparation, while the last two contain the information collected during the field work.
	   \begin{itemize}
	   	\item   a screen shot of the area from Google map, 
	   	\item 	the classification results from the forest assessment \cite{Sumnall2013}, 
	   	\item 	the coloured tree species map and 
	   	\item 	the approximated height map.
	   \end{itemize}
	   
	   
	   \par Comparing the validation dataset created with the classifications done at Bournemouth University (which were not validated) there are misclassification. This is shown in Figures \ref{fig:NF_Area1} and \ref{fig:NF_Area2} and it is likely to occur due to the over-segmentation of trees. Those wrong classifications  justify that validation and field work data are essential for building a good classifier. 
	   
	   \newpage
	   \par The 1st area is included in the LAS file named LDR-FW-FW10\_01-201018715.LAS and it lies inside the limits:  X = (433453 - 433761), Y = (102193 - 102405). The four maps that relates to this areas are shown in Figure \ref{fig:NF_Area1}.
	   
	   \begin{figure} [h!]
	   	\begin{subfigure}[t]{.5\textwidth}
	   		\centering
	   		\includegraphics[width=.9\textwidth]{img/NewForest/Area1GoogleMap}
	   		\caption{Google map screenshot}
	   		\label{fig:Area1GoogleMap}
	   	\end{subfigure} \hfill
	   	\begin{subfigure}[t]{.5\textwidth}
	   		\centering
	   		\includegraphics[width=.9\textwidth]{img/NewForest/Area1Classifications}
	   		\caption{Forest assessment classifications} 
	   		\label{fig:Area1Classifications}
	   	\end{subfigure}
	   	\begin{subfigure}[t]{.5\textwidth}
	   		
	   		\centering
	   		\includegraphics[width=.9\textwidth]{img/NewForest/Area1Fieldwork_Species}
	   		\caption{Tree species map}
	   		\label{fig:Area1Fieldwork_Species}
	   	\end{subfigure} \hfill
	   	\begin{subfigure}[t]{.5\textwidth}
	   		\centering
	   		\includegraphics[width=.9\textwidth]{img/NewForest/Area1Fieldwork_Heights}
	   		\caption{Approximated heights of the trees} 
	   		\label{fig:Area1Fieldwork_Heights}
	   	\end{subfigure}
	   	\caption{The 1st area of interest and the related maps.} %\footnotemark[1].} 
	   	\label{fig:NF_Area1} 
	   \end{figure}
	   
	   
	   
	   \newpage
	   \par The 2nd area is included in the LAS files named LDR-FW-FW10\_01-201018719.LAS and LDR-FW-FW10\_01-201018718.LAS and it lies inside the limits:  X = (436442 - 436835), Y = (102334 - 102585). The four maps created for this areas are shown in Figure \ref{fig:NF_Area2}.
	   
	   \begin{figure} [h!]
	   	\begin{subfigure}[t]{.5\textwidth}
	   		\centering
	   		\includegraphics[width=.9\textwidth]{img/NewForest/Area2GoogleMap}
	   		\caption{Google map screenshot}
	   		\label{fig:Area2GoogleMap}
	   	\end{subfigure} \hfill
	   	\begin{subfigure}[t]{.5\textwidth}
	   		\centering
	   		\includegraphics[width=.9\textwidth]{img/NewForest/Area2Classifications}
	   		\caption{Forest assessment classifications} 
	   		\label{fig:Area2Classifications}
	   	\end{subfigure}
	   	\begin{subfigure}[t]{.5\textwidth}
	   		
	   		\centering
	   		\includegraphics[width=.9\textwidth]{img/NewForest/Area2Fieldwork_Species}
	   		\caption{Tree species map}
	   		\label{fig:Area2Fieldwork_Species}
	   	\end{subfigure} \hfill
	   	\begin{subfigure}[t]{.5\textwidth}
	   		\centering
	   		\includegraphics[width=.9\textwidth]{img/NewForest/Area2Fieldwork_Heights}
	   		\caption{Approximated heights of the trees} 
	   		\label{fig:Area2Fieldwork_Heights}
	   	\end{subfigure}
	   	\caption{The 2nd area of interest and the related maps.} %\footnotemark[1].} 
	   	\label{fig:NF_Area2} 
	   \end{figure}
	   
	   - > Wrong classification due to over segmentation of the trees. 
   \newpage
\section{	Landscape types}
	During the forest assessment in New Forest, not only some validation data were collected, but also useful information about classifying the data. The following images show examples of the five landscape types that were found in New Forest:
	
	1.	Heather fields:
    
      \begin{figure} [!h]
      	\centering
      	\includegraphics[width=\textwidth]{img/NewForest/LT_HeatherFields}
      	\caption{Trees that have been cut down}
      	\label{fig:LT_HeatherFields}
      \end{figure}

    
    2.	Grass with a few scattered trees:
          \begin{figure} [!h]
          	\centering
          	\includegraphics[width=\textwidth]{img/NewForest/LT_GrassWithFewTrees}
          	\caption{Grass with a few scattered trees}
          	\label{fig:LT_GrassWithFewTrees}
          \end{figure}

    3.	Dense Forest:
          \begin{figure} [!h]
          	\centering
          	\includegraphics[width=\textwidth]{img/NewForest/LT_DenseForest}
          	\caption{Dense Forest}
          	\label{fig:LT_DenseForest}
          \end{figure}
          
          \newpage
    
    4.	Bushes and Shrubs
          \begin{figure} [!h]
          	\centering
          	\includegraphics[width=\textwidth]{img/NewForest/LT_BushesAndShrubs}
          	\caption{Trees that have been cut down}
          	\label{fig:LT_BushesAndShrubs}
          \end{figure}

    
    5.	Lakes and rivers, which are more rarely found
          \begin{figure} [!h]
          	\centering
          	\includegraphics[width=\textwidth]{img/NewForest/LT_LakesAndRivers}
          	\caption{Lakes and Rivers}
          	\label{fig:LT_LakesAndRivers}
          \end{figure}

    
    

    \par Please note that the landscape types could significantly differ according to the scanned area. For example, the landscape of New Forest is flat while the landscape of Eaves Wood (another scanned forest in UK) is hilly. The landscape type should be taken into consideration during classifications. 
    
    \subsection{Classification challenges}
    \par This case study further helped into understanding the challenges of creating validation data and writing a tree species classifier. These challenges are listed and explained below with some photos taken during field work:
    
    \par 1.	Field work and remotely sense data collection should happen around the same time to avoid changes that happens over time. In the New Forest case, the airborne data were collected in 2010 and there were some trees have been cut down.
    
    \begin{figure} [!h]
    	\centering
    	\includegraphics[width=\textwidth]{img/NewForest/CC_TreesCutDown}
    	\caption{Trees that have been cut down}
    	\label{fig:CC_TreesCutDown}
    \end{figure}

    
    \par 2.	Machine learning becomes more challenging while the number of classes increases. Regarding tree species classes, it is utopian to include all the tree species. This was justified when the list of tree species used at the tree assessment held by Sumnall \cite{Sumnall2013} didn’t include a number of trees (e.g. holly trees and crabapple), which were widely spread in New Forest. 
    
    \par 3.	There were animals around the forest and it is possible that LiDAR may include laser intensities reflected back by hitting an animal instead branches, leaves and trunks. 
    
        \begin{figure} [!h]
        	\centering
        	\includegraphics[width=\textwidth]{img/NewForest/CC_Animals}
        	\caption{Animals in New Forest}
        	\label{fig:CC_Animals}
        \end{figure}
        
   
    
    \par 4.	Big validation datasets from a single area will not be sufficient, because trees of the same type are usually gathered together. For instance, the 1st selected area has many pine trees while the 2nd one has many oak trees. Therefore, it is important to have many field plots spread well within the area of interest.  
    
    \par 5.	Further, some trees are mixed together which makes it difficult to identify from the data whether they are one or two trees. Examples are shown in Figure \ref{fig:CC_MixedTrees}; on the left image the trunks of the two trees are very close to each other and on the right image a crabapple and an oak tree has grown together. 
    
    \begin{figure} [!h]
    	\centering
    	\includegraphics[width=\textwidth]{img/NewForest/CC_MixedTrees}
    	\caption{Trees, which are mixed together}
    	\label{fig:CC_MixedTrees}
    \end{figure}
    
    \section{Conclusions and Discussion }
    \par To sum up, getting to New Forest was considered essential for better understanding the challenges of remote forest monitoring. During the visit, a small validation dataset was generated; the species and height of trees that are inside the two areas of interest were noted down. Field work is a time consuming task and weeks are required for generating a big enough validation dataset but it is essential for understanding the object of interest (trees) in relation to the scanned data. Challenges identified were also explained and this increased knowledge about forests could lead into implementing a better classifier. 
    
\end{document}
