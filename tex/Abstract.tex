\documentclass{subfiles}

\begin{document}

	\par This study focuses on enhancing the visualisations and classifications of forested areas using coincident full-waveform (fw) LiDAR data and hyperspectral images. The ultimate aim is use both datasets to derive information about forests and show the results on a 3D virtual, interactive environment. Influenced by Persson et al (2005), voxelisation is an integral part of this research. The intensity profile of each full-waveform pulse is accumulated into a voxel array, building up a 3D density volume. The correlation between multiple pulses into a voxel representation produces a more accurate representation, which confers greater noise resistance and it further opens up possibilities of vertical interpretation of the data. The 3D density volume is then aligned with the hyperspectral images using a 2D grid similar to Warren et al (2014) and both datasets are used in visualisations and classifications. 
	\par Previous work in visualising fw LiDAR has used transparent objects and point clouds, while the output of this system is a coloured 3D-polygon representation, showing well-separated structures such as individual trees and greenhouses. The 3D density volume, generated from the fw LiDAR data, is polygonised using functional representation of object (FReps) and the marching cubes algorithm (Pasko and Savchenko, 1994) (Lorensen and Cline, 1987). Further, an optimisation algorithm is introduced that uses integral volumes (Crow, 1984) to speed up the process of polygonising the volume. This optimisation approach not only works on non-manifold object, but also a speed up of up to \verb|51%| was achieved. The polygon representation is also textured by projecting the hyperspectral images into the mesh. In addition, the output is suitable for direct rendering with commodity 3D-accelerated hardware, allowing smooth visualisation. 
	\par In future work, the effects of combining both hyperspectral imagery and fw LiDAR in classifications and visualisations are examined. At first, two pixel wise classifiers, a support vector machine and a Bayesian probabilistic model, will be used for testing the effects of the combination in generating tree coverage maps. Higher accuracy classification results are expected when metrics from both datasets are used together. Regarding the visualisations, the differences of applying surface reconstruction versus direct volumetric rendering will be discussed and an ordered tree structure with integral sums of the node values will be used for speeding up the ray-tracing of direct volumetric rendering and improving memory management of aforementioned optimisation algorithm with integral volumes. Further, deferred rendering is suggested for testing the visual human perception of projecting multiple bands of the hyperspectral images on the FW LiDAR polygon representations. At the end of this project the combination of the datasets will be used along with the watershed algorithm for tree segmentation, which is useful for measuring the stem density of a forest and for tree species classifications. 
	
	


\end{document}