\documentclass{subfiles}

\begin{document}

\par Monitoring forests is important for managing biodiversity, maintaining forest health and noting changes of the canopy structures in order to preserve a resilient ecosystem.  Fieldwork is traditionally used to derive information about forests, but it is a time consuming task and the area that can be surveyed is limited.  For that reason remote sensing was introduced to automate the process and increase the size of the monitored areas. 

\par Airborne LiDAR systems (ALS) are extremely useful in forestry because the scanning laser beam penetrates the tree canopy and collects information from the backscatter off leaves and tree branches down to the ground. Two types of Airborne LiDAR were discussed: discrete and FW LiDAR. The output of the discrete LiDAR is a point cloud which consists of identified peak laser returns. The FW LiDAR systems, for every emitted pulse, record the entire back-scattered signal digitised into equally spaced time intervals, resulting in a set of discretised waveforms. FW LiDAR therefore contains much more information in comparison to discrete but these increased volume of information makes interpretation difficult. 

\par According to NERC-ARF, full-waveform LiDAR acquisitions are frequently requested by clients but there very few research studies that use the data because existing workflows do not support them. For that reason, the overarching aim of this thesis is to make the FW LiDAR data accessible to scientists with a limited background in computing. The open source program DASOS was developed to demonstrate the novel algorithms and structures in this thesis and serves that purpose. The way DASOS handles the FW LiDAR data is fundamentally different from the existing open source tools, and not previously seen in the remote-sensing / forestry literature. Instead of peak points extraction (discrete LiDAR extraction), DASOS voxelises the FW LiDAR data and derives information from the voxelised space; this allows for efficient visualisation and analysis while also allowing multiple laser "shots" passing through a single voxel to reinforce one another, meaning even fainter features can potentially be identified. There are three main functionalities or applications of DASOS, each reflecting a contribution presented in this thesis.

\par The first application is the extraction of an iso-surface from the voxelised FW LiDAR data, which is the essential step to allow for efficient visualisation. The iso-surface extraction was done by introducing Computer Graphics concepts into remote sensing. An algebraic definition of the voxelised FW LiDAR data is defined and the Marching Cubes algorithm is used to extract a set of triangle primitives, which can easily be rendered with commodity 3D accelerated hardware. New data structures were introduced for managing real volumetric data, generated from full-waveform LiDAR data. Optimisations of real volumetric data are difficult to manage because they neither manifold or closed. While seeking for an algorithmic approach for optimising iso-surface extraction, we test the performance of six different data structures, of which three of them are new. The new structures are: 1) Integral Volumes, which is an extension of Integral Images into 3D and allows identification of large, potentially non-cubic, empty areas in constant time ; 2) Octree Max and Min, where the minimum and maximum values are stored into the branch nodes of an octree. Additionally, a logarithmic approach for efficiently finding neighbouring voxels is introduced; 3) The Integral Tree, which is an attempt to combine Integral Volumes with octrees. Overall, Integral Volumes perform the fastest of all the tested approaches, but they require that all the empty voxels are stored in memory. Depending on the size of the data and CPU memory available, applicability of the approaches varies.

\par The second application aims to enhance the visualisations and classifications of airborne remote sensing data by aligning hyperspectral and FW LiDAR data; an approach that could be extended to merge other acquisition modalities, such as radar or thermal imagery.  In order to preserve the highest possible quality of the hyperspectral data, their original sense of geometry (`level 1') is used to reduce blurring. A spatial representation of the pixel geo-locations is stored within a hashed table with buckets of points to quickly find the nearest geo-location of a hyperspectral pixel to a point (i.e. a vertex for the polygonal meshes or the centre of each column within the voxelised FW LiDAR for generating aligned metrics). There are two outputs of this application: the coloured polygonal meshes, which are easier to view and visually interpret, and the tree coverage maps, which are generated using a Bayesian probabilistic model.  Due to the combination of the data, higher accuracy classification results are obtained.

\par The last application is the detection of dead standing eucalyptus trees from a native Australian forest without tree delineation. Shortage of hollows available for wildlife colonisation has been predicted and hollows are more likely to exist on aged dead trees, making rapid and automated identification a priority. Tree delineation is difficult because there are multiple trunk splits and the resolution of the cost-effective acquired data does not enable bottom to top delineation. Dead tree detection without tree delineation is a new research direction; this thesis showed that it is possible, although the results are at an early stage and there is room for improvement. The methodology uses feature vectors from DASOS to characterise dead and alive trees. The random forest is used to identify the most significant features. Then an image indicating the probability of a pixel to be from a dead or a alive tree is calculated using a distance-weighted k-nn algorithm. The position of the fieldplots is estimated using a seed growth algorithm to segment different dead trees after the ground and alive trees are removed from the image. The results showed that the increased amount of training samples do not improve the results, because of the increased amount of noise included within the training samples; a fault in the available data more than the approach.  Additionally, the cylindrical shape used to extract features is more meaningful in comparison to the cuboid. This approach clearly performs better than random prediction but, as it is early research, improvements could be applied (e.g. categorising feature vectors according to tree heights). 


{\color{blue}
\par While this thesis eases handling of huge datasets of FW LiDAR and hyperspectral imagery for forestry applications, there are still many improvements possible in future work, such as:
\begin{itemize}
	\item Create an interactive and friendly graphical user interface of DASOS. This thesis focuses on the algorithms that optimise handling the data. For that reason, the user interface of DASOS is scripting and there are no interactive features (i.e. being able to measure the distance between two trees). Nevertheless, the most likely people to use the software are geographers and foresters with no or very little background in scripting and computer science. For that reason, a friendlier and interactive environment in DASOS will significantly increase the usage of FW LiDAR data.  
	\item Another addition is to test whether direct rendering or surface reconstruction of the FW LiDAR data works more efficiently in interactive and visual environments. So far, the surface reconstruction approach was tackled. Investigating how well direct rendering of FW LiDAR works is another research direction. {\color{pink}You mean directly plotting each sample as a point?  Might be something that spdlib tried in its point viewer? http://www.spdlib.org/doku.php#spd_3d_points_viewer }
	\item In the southern UK, climate change could significantly affects the health of forests due to native tree species being unable to adapt to the increased severity and frequency of drought during the summer period, while the possibility of increased insect pests and tree disease is high ~\cite{Read2009}.  Since 2015, NERC ARF provides thermal hyperspectral imaging from the Long Wavelength Infrared (LWIR) bands of the electromagnetic spectrum collected from the Specim AisaOWL sensor. The change of heat within a forest is an indicator for assessing forest health. DASOS could be extended to support LWIR bands and the combination of the structural information of the FW LiDAR and the thermal parameters could improve forest health assessment in UK.  	
	\item Regarding dead tree detection, it was shown that it could be done without tree delineation. Nevertheless, the range of heights within the forest is large. For that reason, the precision and recall of the results could be improved by categorising the trees according to their heights and generating training feature vectors relevant to the height of each category. 
	
\end{itemize}



\par 


\par 
}
RESOUNDING LAST SENTENCE SAYING THAT YOU MADE NOVEL CONTRIBUTIONS TO HOW TO DEAL WITH FULL WAVEFORM DATA IN A VOXEL WORLD.  GIMME A DOCTORATE NOW.


\end{document}
