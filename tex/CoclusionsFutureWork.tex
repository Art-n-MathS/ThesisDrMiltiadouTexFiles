\documentclass{subfiles}

\begin{document}

\par Monitoring forests is important for managing biodiversity, forest health and changes of the canopy structures in order to preserve a resilient ecosystem. Fieldwork is traditionally used to derive information about forest, but it is a time consuming task. For that reason remote sensing was introduced to automated the process and increase the size of the monitored areas. 

\par ALS systems are extremely useful in forestry because their laser beam penetrates the tree canopy and collects information from between the tree branches. Two types of Airborne LiDAR were discussed: the discrete and FW LiDAR. The output of the discrete LiDAR is a points cloud which is consisted from peak laser returns. The FW LiDAR systems, for every emitted pulse, it records the entire back-scattered signal digitised into equally spaced time intervals. Therefore the FW LiDAR is a set of discrete waveforms. FW LiDAR contains much more information in comparison to discrete but the increase amount of information makes interpretation of the data difficult. 

\par According to NERC-ARF, full-waveform LiDAR are requested by clients but there very few researches that uses the data because existing workflows do not support them. For that reason, the overarching aim of this thesis is to make the FW LiDAR data accessible to scientist with limited background in computing. The open source program DASOS was develop and serves that purpose. The way DASOS handles the FW LiDAR data is fundamentally different from the existing open source tools. Instead of peak points extraction (discrete LiDAR extraction), DASOS voxelises the FW LiDAR data and derives information from the voxelised space. There are three main functionalities of DASOS and for each one of them an application was presented in this thesis.  

\par The first application is the extraction of an iso-surface from the voxelised FW LiDAR data. The iso-surface extraction was done by introducing Computer Graphics concepts into remote sensing. An algebraic definition of the voxelised FW LiDAR data is defined and the Marching Cubes algorithm is to extract a set of triangle primitives, which can easily be rendered with commodity 3D accelerated hardware. New data structures was further introduced for managing real volumetric data, generated from full-waveform LiDAR data. Optimisations of real volumetric data are difficult to manage because they neither manifold or closed. While seeking for an algorithmic approach for optimising iso-surface extraction, we test the performance of six different data structures, of which three of them are new. The new structures are: 1. The Integral Volumes, which is an extension of Integral Images into 3D and allows identification of big empty areas that are not necessarily cubic in constant time. 2. Octree Max and Min, where the minimum and maximum values are stores into the branch nodes of an octree. Additionally, an logarithmic approach for efficiently finding neighbouring voxels is introduced. 3. The Integral Tree, which is an attempt to combine Integral Volumes with octrees. Overall, Integral Volumes perform the fastest of all the tested approaches, but they required storing all the empty voxels into memory. Depending on the size of the data and CPU memory available, applicability of the approaches varies. 

\par The second application aims to enhance the visualisations and classifications of airborne remote sensing data by aligning hyperspectral and FW LiDAR data.  In order to preserve the highest possible quality of the hyperspectral data, their original sense of geometry (`level 1') is used to reduce blurring. A spatial representation of the pixel geo-locations is stored within a hashed table with bucket points to quickly find the nearest geo-location of a hyperspectral pixel to a point (i.e. a vertex for the polygonal meshes or the centre of each column within the voxelised FW LiDAR for generating aligned metrics). There are two outputs of this application: the coloured polygonal meshes and the tree coverage maps, which are generated using a Bayesian probabilistic model and due to the combination of the data, higher accuracy classification results are produced. 




\par The last application is the detection of dead standing eucalyptuses from a native Australian forest without tree delineation. Shortage of hollows available for colonisation has been predicted and hollows are more likely to exist on aged dead trees. Tree delineation is difficult because there are multiple trunk splits and the resolution of the cost effective acquired data does not enable bottom to top delineation. Dead tree detection without tree delineation is a new research direction and we showed that it is possible to be done but it is on earlier stages and there is room for improvement. The methodology uses feature vectors from DASOS to characterise dead and alive trees. The random forest is used to identify the most significant features. Then an image indicating the probability of a pixel to be from a dead or a alive tree is calculated using a distance-weighted k-nn algorithm. The position of the fieldplots is estimated using a seed growth algorithm to segment different dead trees after the ground and alive trees are removed from the image. The results showed that the increase amount of training samples do not improve the results, because of the increased amount of noise included within the training samples. Additionally, the cylindrical shape used to extract features is more meaningful in comparison to the cuboid. By the end, this approach clearly performs better than random prediction but as it is in early research stages improvements could be applied (e.g. categorising feature vectors according to tree heights). 



\end{document}