
\documentclass{subfiles}

\begin{document}

 
 
 






\par 

and by using those representations DASOS can produce 3D polygons and metrics useful in deriving information about the scanned areas.

\par :

\par Firstly, a new and fast way of aligning the FW LiDAR with Remotely Sensed Images has been developed in DASOS and by generating tree coverage maps it was shown that the combination of those datasets confers better remote survey results.  This work was presented at the 36th ISRSE International Conference. 

\par Secondly, automated detection of dead trees in native Australian forests has a significant role in protecting animals, which live in those trees and are close to extinction. DASOS allow the generation of 3D signatures characterising dead trees. A comparison between the discrete and FW LiDAR is performed to demonstrate the increased survey accuracy obtained when the FW LiDAR are used. 

\par Finally, the last application is for improving visualisations for foresters. Foresters have a great knowledge about forests and can derive a wealth of information directly from visualisations of the remotely sensed data. This reduces the travelling time and cost of getting into the forests. This research optimises visualisations by using the new FW LiDAR representations and a speed of up to 51\verb|%| has been achieved.

\par FW LiDAR has great potentials in forestry and this research has already started to have an impact in the FW LiDAR community by making those huge datasets easier to handle. DASOS is now used at Interpine Group Ltd, a world leading Forestry Company in New Zealand and it has been tested from a PhD student at Bournemouth University who looks into estimating bird distribution in the New Forest. In the future, it is expected that DASOS will be widely used in remote forest surveys (i.e. estimating the commercial value of a forest and detecting infected trees at early stages for treatment). 


\end{document}


