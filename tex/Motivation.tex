
\documentclass{subfiles}

\begin{document}

Importance of Forest Monitoring and Applications

 - Forest Health: Detecting tree diseases at early stages and treat trees 
 Potential protection of vegetation from diseases and pests ~\cite{ForestryCommision2014}.  
 An example of pests, which destroy vegetation, is the Brushtail Possums, which were initially brought to New Zealand for fur trade, but they have escaped and they are a thread to native forests and vegetation ~\cite{DepartementOfConversation2014}. Measuring the effect on the forest health is difficult. The Landcare Research and Department of Conservation (DOC), in cooperation with Aberystwyth University, collected full-waveform LiDAR data from the North Island of New Zealand to investigate those effects. But according to Dr Pete Bunting, who is involved in that research, there are no available tools to those organisations for reading and processing full-waveform LiDAR data (personal communication). 
 
 - Biodiversity Management - Tree hollows 
 - Estimating the value of a commercial forest - wood trade
 - Archaeology: structure below dense forest lead into the detection of ancient hidden city in Cambodian jungle ~\cite{Lawrie2014}
 - Woodland Phenology: by observing the change in number of trees species, the recurring of natural phenomena can be predicted and potentially pre`vented ~\cite{BBC2009}. 
 
 
 
Fieldwork - time consuming



\par LiDAR refers to the acquisition of information from laser scanners. There are two types of LiDAR data, the discrete and the full-waveform (FW). The discrete LiDAR measures the round trip distance time of the laser beam and calculates the position of a few hit points, while the FW LiDAR records the entire backscattered signal. But the increased amount of information recorded within the FW LiDAR makes handling of those data difficult. As an indication, a 9.3GB discrete LiDAR from New Forest, UK, corresponds to 55.7GB of FW LiDAR. The discrete LiDAR has been widely used and a 40\verb|%| reduction of fieldwork has been achieved at Interpine Ltd Group, New Zealand, with that technology. Regarding the FW LiDAR, scientists understand their concepts and potentials but due to the lack of available tools able to handle these large datasets, there are very few uses of FW LiDAR ~\cite{Anderson2015}.
, which aims to break the barrier between understanding and using the FW LiDAR.

\par 

and by using those representations DASOS can produce 3D polygons and metrics useful in deriving information about the scanned areas.

\par :

\par Firstly, a new and fast way of aligning the FW LiDAR with Remotely Sensed Images has been developed in DASOS and by generating tree coverage maps it was shown that the combination of those datasets confers better remote survey results.  This work was presented at the 36th ISRSE International Conference. 

\par Secondly, automated detection of dead trees in native Australian forests has a significant role in protecting animals, which live in those trees and are close to extinction. DASOS allow the generation of 3D signatures characterising dead trees. A comparison between the discrete and FW LiDAR is performed to demonstrate the increased survey accuracy obtained when the FW LiDAR are used. 

\par Finally, the last application is for improving visualisations for foresters. Foresters have a great knowledge about forests and can derive a wealth of information directly from visualisations of the remotely sensed data. This reduces the travelling time and cost of getting into the forests. This research optimises visualisations by using the new FW LiDAR representations and a speed of up to 51\verb|%| has been achieved.

\par FW LiDAR has great potentials in forestry and this research has already started to have an impact in the FW LiDAR community by making those huge datasets easier to handle. DASOS is now used at Interpine Group Ltd, a world leading Forestry Company in New Zealand and it has been tested from a PhD student at Bournemouth University who looks into estimating bird distribution in the New Forest. In the future, it is expected that DASOS will be widely used in remote forest surveys (i.e. estimating the commercial value of a forest and detecting infected trees at early stages for treatment). 


\end{document}


