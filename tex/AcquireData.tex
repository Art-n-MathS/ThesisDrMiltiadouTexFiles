\documentclass{subfiles}

\begin{document}

two organisation:
	1. Natural Environment Research Council’s Airborne Research and Survey Facility (NERC ARSF)
	2. Interpine Ltd Group () 
	
	1. used for the hyperspectral Alignment
	2. dead tree detection
	both for visualisations and optimisations of various data structures
	
	\subsection{Instrumenets}
	1. Instruments 
		- Leica ALS50-II
		- AISA Eagle and Hawk
		
	2. Instrument
		- Riegl LMS-Q680i
		
	

	The sample data used for this thesis are provided by Natural Environment Research Council’s Airborne Research and Survey Facility (NERC ARSF). The datasets mainly used were scanned on the 8th of April in 2010 at New Forest in UK. For every flightline, two Airborne Remote Sensing datasets are given: 
	
	-	Full-waveform (FW) LiDAR data collected from the Leica ALS50-II system
	-	Hyperspectral Images collected from the AISA Eagle and Hawk instruments
	
	For each flightline, there is a FW LiDAR file, as well as the corresponding hyperspectral data. But since the data are collected from different instruments they are not aligned (see section 5 for more information on aligning the data). 
	
	In the following sub-sections, background information on the input datasets is given, starting from simple concepts like Remote Sensing. Then, the inputs are explained in practise; what the actual input data are and how they are interpreted. The challenges of dealing with the data are also outlined and discussed. 



	\subsection{Full-waveformLiDAR}
	


	
	\subsection{Hyperspectral Imagery}
	
	
\end{document}