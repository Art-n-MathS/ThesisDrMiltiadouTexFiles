\documentclass{subfiles}

\begin{document}

The overarching aim of future work is to improve and optimise visualisations of full-waveform LiDAR data and hyperspectral images for remote forest surveying.

Objectives:\newline
1.	Enable browsing of very large scale datasets with many points and spectral bands in an efficient manner
2.	Estimate tree coverage and investigate the potential of integrating multiple remote sensing datasets in forestry
3.	Tree crown detection and height estimations in comparison to human detection, which will benefits wood trade and forest management
4.	Enable experts to establish wood coverage 
5.	Enable understanding of forestry concepts through 3D visualisations
6.	Investigate data structures that will be better for volumetric rendering and efficient management of large point clouds

This project explores visualisation and data-understanding for full waveform LIDAR systems. 

- How can large full waveform datasets be effectively and efficiently visualised? (especially in combination with other forms of data, such as hyperspectral images).
- How can terrain classification systems can be modified to effectively make use of full-waveform data? (for example, detection of dead trees for protecting animals living inside dead trees in Australia).  
- Can visualisation and classification be improved by inference of high quality 3D information, for example, using priors over the space of 3D elements and compatibility between multiple observations?
The project will also encompass other facets of large volume environmental dataset visualisation and understanding as appropriate.




\end{document}