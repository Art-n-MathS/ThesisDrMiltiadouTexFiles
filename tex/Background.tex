\documentclass{subfiles}

\begin{document}
	\par Remote sensing refers to acquisition of information about objects, like vegetation and archaeological monuments, without physical contact and the interpretation of that information.  The sensors used to capture the information are divided into passive and active. For example satellite photography is passive because information are collected from the sun light, while Airborne Laser Scanners (ALS) are active because they emit laser beams and collects information from the backscattered laser energy~\cite{Smith2012}.
	
	\par According to Wanger et al, Airborne Laser Scanning (ALS) is a growing technology used in environmental research to collect information about the earth like vegetation and tree species. Comparing ALS with traditional photography, ALS are not influenced by light and therefore are less dependant to weather conditions (ie. they collect information from below clouds). The laser beam further penetrates the tree canopies recording this way information about the forest structure below the canopy, as well as the ground ~\cite{Wanger2004}. ALS are divided into pulse systems and continuous wavelength systems. Continuous wavelength systems are more complicated, because they obtain one extra physical parameter. Further, according to Wehr and Lohr continuous wavelength Systems are 85 times less accurate than pulse systems ~\cite{Wehr1999}.
	
	\par LiDAR (Light Detection And Ranging) are passive ALS that acquire information by repeatedly emitting pulses ~\cite{Wehr1999}. The LiDAR systems are either small-footprint or large-footprint. On the one hand, small-footprint systems record higher resolution but cannot guarantee that the pulse will reach the ground, making height related calculations difficult. On the other hand, large-footprint scanners have wider diameters and can therefore scan wider areas with the likelihood of recording the ground to be higher ~\cite{Mallet2009}. 
	
	\par In addition, there are two types of LiDAR data, the discrete and the full-waveform (FW). The discrete LiDAR measures the round trip distance time of the laser beam and records the position of a few hit points, while the FW LiDAR stores the entire backscattered signal. The discrete LiDAR has been widely used and a 40\verb|%| reduction of fieldwork has been achieved at Interpine Ltd Group, New Zealand, with that technology. 
	
	\par Regarding the FW LiDAR, scientists understand their concepts and potentials but due to the lack of available tools able to handle these large datasets, there are very few uses of FW LiDAR ~\cite{Anderson2015}. The design of the first FW LiDAR system was introduced in 1980s, but the first operational system was developed by NASA in 1999~\cite{Chauve2007}. The increased amount of information recorded within the FW LiDAR suggests many new possibilities and problems from the point of view of image understanding, remote surveying and visualisation. As an indication, a 9.3GB discrete LiDAR from New Forest, UK, corresponds to 55.7GB of FW LiDAR. The most common approach of interpreting the FW LiDAR is the Gaussian decomposition of the waveforms and extraction of peak points ~\cite{Wanger2006}. Neunschwander et al used this approach for Landcover classification while Reitberger et al applied it for distinguishing deciduous trees from coniferous trees ~\cite{Neuenschwander2009}~\cite{Reitberger2008}. Chauve et al further proposed an approach of improving the Gaussian model in order to increase the density of the points extracted from the data and consequently improve point based classifications applied on FW LiDAR data ~\cite{Chauve2007}. 
	
	\par While echo decomposition identifies significant features as points, the full waveform data also contains information in single shots that may be below the significance threshold. The waveform samples data can be accumulated from multiple shots into a voxelised volume. Voxelisation was firstly introduced by Persson et al, who used voxelisation to visualise the waveforms using different transparencies \cite{Persson2005} and it seems to be the future of FW LiDAR data with the literature to moving toward that direction. In 2016, Cao et al used it for tree species identification \cite{Cao2016} and Sunmall et al characterised forest canopy from a voxelised vertical profile \cite{Sumnall2016}. This innovative approach is an integral part of this thesis and it is used for both visualisations and classifications \cite{Miltiadou2014}\cite{Miltiadou2015}. 
	
		
	 
	
    
\end{document}