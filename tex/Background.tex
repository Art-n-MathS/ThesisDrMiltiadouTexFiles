\documentclass{subfiles}

\begin{document}
	\par Full-waveform LiDAR data are airborne remote sensing data. Remote sensing refers to acquisition of information about objects, like vegetation and archaeological monuments, without physical contact and the interpretation of that information.  The sensors used to capture the information are divided into passive and active. For example satellites are passive sensors because they collect information from the sun light, while Airborne Laser Scanners (ALS) are active because they emit a laser pulse and collects information from its returns ~\cite{Smith2012}.
	
	\par According to Wanger et al, Airborne Laser Scanning (ALS) is a growing technology used in environmental research to collect information about the earth like vegetation and tree species. Comparing ALS with traditional photography scanning, ALS is more flexible because it is not influenced by light and it can collect information from below the tree canopies ~\cite{Wanger2004}. ALS are divided into pulse systems and continuous wavelength systems. Continuous wavelength systems are more complicated, because they obtain one extra physical parameter. Further, according to Wehr and Lohr continuous wavelength Systems are 85 times less accurate than pulse systems ~\cite{Wehr1999}.
	
	\par Laser radars are ALS systems that are able to obtain range images by repeatedly emitting pulses. Those systems are commonly known as LiDAR (Light Detection and Ranging) ~\cite{Wehr1999}. Full-waveform LiDAR can encode and store the entire radiation returned from each pulse, digitised over time. The design of the first FW LiDAR system was introduced in 1980s, but the first operational system was developed by NASA in 1999 ~\cite{Chauve2007}.  There are two types of LiDAR systems: small-footprint and large-footprint. On the one hand, small-footprint returns a higher detailed accuracy but it cannot guarantee that the pulse will reach the ground, making height related calculations difficult. On the other hand, since the diameter of large-footprint scanners is much bigger, they are able to scan wider areas and the last target is much more likely to be the ground ~\cite{Mallet2009}.   
	
	\par The most common approach of interpreting the FW LiDAR from the Leica instrucment is to decompose the waveform into a sum of Gaussian functions, each representing a peak return, and sequentially extract point clouds from ~\cite{Wanger2006}. Neunschwander et al (2009) used this approach for Landcover classification while Reitberger et al (2006) applied it for distinguishing deciduous trees from coniferous trees ~\cite{Neuenschwander2009} ~\cite{Reitberger2008}. In 2007, Chauve et al proposed an approach of improving the Gaussian model in order to increase the density of the points extracted from the data and consequently improve point based classifications applied on FW LiDAR data ~\cite{Chauve2007}. 
	
	\par Nevertheless, the state-of-art sensors, RIEGL LMS-Q780 and RIEGL LMS-Q680i are native full-waveform sensors. Therefore, the discrete LiDAR are produced by extracting peak points at post-processing. The Gaussian decomposition used to extract points from full-waveform at the aforementioned papers ~\cite{bibid} is now used  before the data are delivered. Therefore the concept of extracting a denser point clouds using Gaussian decomposition does not apply on data from Riegl. That was proved by extracting peak points from RIEGL FW LiDAR data using the pulseextract from LAStools ~\cite{LAStools}. The number of points extracted was the same as the number of points existed inside the discrete LiDAR files delivered.

	
	\par new ways of interpreting full-waveform - Cao voxelisation seems the future
	
	~\cite{Cao2016} 
	
    \par To sum up 
    
\end{document}