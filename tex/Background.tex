\documentclass{subfiles}

\begin{document}
	\par Remote sensing refers to the acquisition of information about objects, {\color{blue} for example} vegetation and archaeological monuments, without physical contact and the interpretation of that information.  The sensors used to capture the information are divided into passive and active. For example satellite photography is passive because information are collected from the {\color{blue} reflected natural} sun light, while Airborne Laser Scanners (ALS) are active because they emit laser beams and collects information from the backscattered laser energy~\cite{Smith2012}.
	
	\par According to Wanger et al, Airborne Laser Scanning (ALS) is a growing technology used in environmental research to collect information about the earth like vegetation and tree species. Comparing ALS with traditional photography, {\color{blue} ALS is not influenced by light and it is therefore less dependant on weather conditions (ie. it collects information from below the clouds)}. The laser beam further penetrates the tree canopies {\color{blue} allowing it to record} information about the forest structure below the canopy, as well as the ground ~\cite{Wanger2004}. {\color{blue}ALS methods are divided into pulse systems, which repeatedly emit pulses, and continuous wavelength systems that continuously emit light. They both acquire information from the backscattered laser intensity over time, but continuous wavelength systems are more complicated because they obtain one extra physical parameter, the frequency of the ranging signal}. Further, according to Wehr and Lohr, continuous wavelength systems are 85 times less accurate than pulse systems ~\cite{Wehr1999}.
	
	\par {\color{blue} LiDAR (Light Detection And Ranging) systems are passive and pulse laser scanning systems ~\cite{Wehr1999}. They are divided into two groups according to the diameter of the footprint left by the laser beam on the ground. This diameter depends on the beam divergence and the distance between the sensor and the target. The small-footprint systems have a 0.2-3m diameter, have been widely commercialised and are mostly carried on planes (ALS systems). In contrast, the large-footprint systems have a wider diameter (10-70m) and during experiments they were mostly adjusted on satellites. Small-footprint systems record at higher resolution but it cannot guarantee that every pulse will reach the ground due to the small diameter of their footprint, making topographic measurements difficult. In contrast, large-footprint scanners have wider diameters and can therefore scan wider areas with the likelihood of recording the ground to be higher ~\cite{Mallet2009} }. 
	
	\par In addition, there are two types of LiDAR data, the discrete and the full-waveform (FW). {\color{blue} The discrete LiDAR records a few peaks of the reflected laser intensity, while the FW LiDAR stores the entire backscattered signal}. The discrete LiDAR has been widely used and a 40\verb|%| reduction of fieldwork has been achieved at Interpine Ltd Group, New Zealand, with that technology. Regarding the FW LiDAR, scientists understand their concepts and potentials but due to the shortage of available tools able to handle these large datasets, there are very few uses of FW LiDAR ~\cite{Anderson2015}. 
	
	\par  The design of the first FW LiDAR system was introduced in 1980s, but the first operational system was developed by NASA in 1999~\cite{Chauve2007}. The increased amount of information recorded within the FW LiDAR suggests many new possibilities and problems from the point of view of image understanding, remote surveying and visualisation. As an indication, a 9.3GB discrete LiDAR from New Forest, UK, corresponds to 55.7GB of FW LiDAR. 
	
	{\color{blue} This research is focused on the representations of the FW LiDAR data and contributes in both forestry visualisations and classifications. Two datasets are used for testing and evaluation: the New Forest and the RedGum dataset. An in depth explanation of LiDAR systems and the specifications, differences and challenges of the two dataset are given in  and  Section \ref{AcquireData}. An overview of the thesis along with its aims, objectives and contributions are then outlined at Section \ref{PipeLine}. }
	    
\end{document}